\PassOptionsToPackage{svgnames,dvipsnames}{xcolor}

\documentclass[12pt]{cmuthesis}

\usepackage[Lenny]{fncychap}
\ChNameVar{\Large}

\usepackage[%
colorlinks=true,allcolors=link_color,pageanchor=true,%
plainpages=false,pdfpagelabels,bookmarks,bookmarksnumbered,%
]{hyperref}

\usepackage[style=alphabetic,natbib=true,backend=biber,maxnames=10]{biblatex}
\bibliography{refs.bib}

\usepackage{totcount}
\newtotcounter{citenum}
\AtEveryBibitem{\stepcounter{citenum}}

\DeclareFieldFormat{citehyperref}{%
  \DeclareFieldAlias{bibhyperref}{noformat}% Avoid nested links
  \bibhyperref{#1}}

\DeclareFieldFormat{textcitehyperref}{%
  \DeclareFieldAlias{bibhyperref}{noformat}% Avoid nested links
  \bibhyperref{%
    #1%
    \ifbool{cbx:parens}
      {\bibcloseparen\global\boolfalse{cbx:parens}}
      {}}}

\savebibmacro{cite}
\savebibmacro{textcite}

\renewbibmacro*{cite}{%
  \printtext[citehyperref]{%
    \restorebibmacro{cite}%
    \usebibmacro{cite}}}

\renewbibmacro*{textcite}{%
  \ifboolexpr{
    ( not test {\iffieldundef{prenote}} and
      test {\ifnumequal{\value{citecount}}{1}} )
    or
    ( not test {\iffieldundef{postnote}} and
      test {\ifnumequal{\value{citecount}}{\value{citetotal}}} )
  }
    {\DeclareFieldAlias{textcitehyperref}{noformat}}
    {}%
  \printtext[textcitehyperref]{%
    \restorebibmacro{textcite}%
    \usebibmacro{textcite}}}


\usepackage{fullpage}
\usepackage{graphicx}
\usepackage{amsmath}
\definecolor{link_color}{RGB}{0,128,255}

\usepackage[%
letterpaper,twoside,vscale=.8,hscale=.75,nomarginpar,hmarginratio=1:1
]{geometry}

\usepackage{graphicx} % more modern
\usepackage{subfigure}

\usepackage{todonotes}
\newcommand{\todon}[1]{\todo[color=red!40,inline,size=\small]{TODO: #1}}
\newcommand{\todoc}{\todo[color=red!40,inline,size=\small]{TODO: Complete}}

\usepackage{amsmath}
\usepackage{amssymb}
\usepackage{amsthm}
\usepackage{arydshln}


\usepackage{accents}
\newcommand{\ubar}[1]{\underaccent{\bar}{#1}}

\usepackage{stackengine}

\usepackage{wrapfig}

\newtheorem{proposition}{Proposition}
\newtheorem{assumption}{Assumption}
\newtheorem{theorem}{Theorem}
\newtheorem{corollary}{Corollary}
\newtheorem{lemma}[theorem]{Lemma}

% \MakeRobust{\Call}
\newcommand*\Let[2]{\State #1 $\gets$ #2}

\definecolor{lightgray}{gray}{0.95} % 10%

\usepackage{hyperref}
\newcommand{\theHalgorithm}{\arabic{algorithm}}


\usepackage{easytable}

\usepackage[capitalise,nameinlink,noabbrev]{cleveref}

\usepackage{stmaryrd}

\usepackage{algorithm}
\usepackage{algorithmicx}
\usepackage{algpseudocode}
\algnewcommand{\LeftComment}[1]{\Statex \(\triangleright\) #1}

\newcounter{module}
\makeatletter
\newenvironment{module}[1][htb]{%
  \let\c@algorithm\c@module
    \renewcommand{\ALG@name}{Module}%
   \begin{algorithm}[#1]%
  }{\end{algorithm}}
\makeatother
\crefname{module}{Module}{Modules}

\usepackage{booktabs}

\usepackage{caption}

\usepackage{listings,textcomp,color}
\definecolor{backcolour}{rgb}{0.95,0.95,0.92}
\definecolor{deepblue}{rgb}{0,0,0.5}
\definecolor{deepred}{rgb}{0.6,0,0}
\lstset{language=Python,upquote=true,
  basicstyle=\ttfamily\footnotesize,
  commentstyle=\textit,stringstyle=\upshape,
  numbers=left,numberstyle=\footnotesize,stepnumber=1,numbersep=5pt,
  backgroundcolor=\color{backcolour},frame=single,tabsize=2,
  showspaces=false,showstringspaces=false,showtabs=false,
  breaklines=true,breakatwhitespace=true,escapeinside=||,
  emph={cp, torch, cpth},emphstyle=\color{deepred},
  keywordstyle=\color{deepblue},
}

% Python style for highlighting
% \DeclareFixedFont{\ttm}{T1}{txtt}{m}{n}{12}  % for normal
% \definecolor{deepgreen}{rgb}{0,0.5,0}
% \lstset{
% language=Python,
% basicstyle=\ttm,
% otherkeywords={self},             % Add keywords here
% keywordstyle=\ttb\color{deepblue},
% emph={cp},          % Custom highlighting
% emphstyle=\ttb\color{deepred},    % Custom highlighting style
% stringstyle=\color{deepgreen},
% frame=tb,                         % Any extra options here
% showstringspaces=false            %
% }

\usepackage{xspace}

\usepackage{framed}

%%% Local Variables:
%%% mode: latex
%%% TeX-master: "main"
%%% End:

\DeclareMathOperator*{\argmax}{argmax}
\DeclareMathOperator*{\argmin}{argmin}
\DeclareMathOperator*{\diag}{diag} \DeclareMathOperator*{\tr}{tr}
\DeclareMathOperator*{\maximize}{maximize}
\DeclareMathOperator*{\minimize}{minimize}
\DeclareMathOperator*{\st}{s.t.}
\DeclareMathOperator*{\subjectto}{subject\;to}
\DeclareMathOperator*{\vect}{vec} \DeclareMathOperator*{\mat}{mat}
\DeclareMathOperator{\prox}{prox}
\DeclareMathAlphabet\mathbfcal{OMS}{cmsy}{b}{n}

\newcommand{\I}{\mathcal{I}}
\newcommand{\J}{\mathcal{J}}
\newcommand{\RR}{\mathbb{R}}
\newcommand{\R}{\mathbb{R}}
\newcommand{\dd}{\mathsf{d}}
\newcommand{\DD}{\mathsf{D}}

% \newcommand{\nwc}{\newcommand}
% \DeclareMathOperator*{\maximize}{maximize}
% \DeclareMathOperator{\prox}{prox}
% \DeclareMathOperator*{\argmin}{argmin}
% \DeclareMathOperator*{\argmax}{argmax}
% \DeclareMathOperator*{\minimize}{minimize}
% \DeclareMathOperator*{\subjectto}{subject\;to}
% \DeclareMathOperator*{\st}{s.t.}

% \newcommand{\uu}{\bm{u}}
% \newcommand{\U}{\mathcal{U}}
% \newcommand{\fix}{\marginpar{FIX}}
% \newcommand{\new}{\marginpar{NEW}}
% \newcommand{\x}{\bm{x}}
% \newcommand{\X}{\mathcal{X}}
\newcommand{\D}{\mathcal{D}}
\newcommand{\X}{\mathcal{X}}
\newcommand{\Y}{\mathcal{Y}}
% \newcommand{\s}{\bm{s}}
% \newcommand{\aaa}{\bm{a}}
% \newcommand{\mmu}{\bm{\mu}}
\newcommand{\E}{\mathbb{E}}
% \newcommand{\f}{\bm{f}}
% \newcommand{\F}{\bm{F}}
% \newcommand{\kk}{\bm{k}}
% \newcommand{\PP}{\bm{P}}
% \newcommand{\vv}{\bm{v}}
% \newcommand{\MM}{\bm{M}}
\newcommand{\LL}{\mathcal{L}}
\newcommand{\JJ}{\mathcal{J}}
\newcommand{\ZZ}{\mathbb{Z}}

\newcommand{\xinit}{x_{\rm init}}
\newcommand{\uinit}{u_{\rm init}}
\newcommand{\ustar}{{u^\star}}
\newcommand{\vstar}{{v^\star}}
\newcommand{\sstar}{{s^\star}}
\newcommand{\xstar}{{x^\star}}
\newcommand{\ystar}{{y^\star}}
\newcommand{\zstar}{{z^\star}}

\newcommand{\CC}{\mathcal{C}}
\newcommand{\K}{\mathcal{K}}
% \newcommand{\RR}{\mathbb{R}}
% \newcommand{\ZZ}{\mathbb{Z}}
\newcommand{\Res}{\mathcal{R}}

\newcommand{\menge}[2]{\big\{{#1}~\big |~{#2}\big\}}

\newcommand{\eg}{{\it e.g.}\xspace}
\newcommand{\ie}{{\it i.e.}\xspace}

\newcommand{\LQR}{\ensuremath{\mathrm{LQR}}}
\newcommand{\MPC}{\ensuremath{\mathrm{MPC}}}

\newcommand{\LML}{\ensuremath{\mathcal{L}}}
\newcommand{\cvxpy}{\texttt{cvxpy}\xspace}
\newcommand{\qpth}{\texttt{qpth}\xspace}

\newcommand{\cblock}[3]{
  \hspace{-1.5mm}
  \begin{tikzpicture}
    [
    node/.style={square, minimum size=10mm, thick, line width=0pt},
    ]
    \node[fill={rgb,255:red,#1;green,#2;blue,#3}] () [] {};
  \end{tikzpicture}%
}

\newcommand\cmax{\textsc{Cmax}}
\newcommand\cmaxpp{\textsc{Cmax++}}
\newcommand{\Qbound}{\mathcal{Q}}

\newcommand\statespace{\mathbb{S}}
\newcommand\actionspace{\mathbb{A}}
\newcommand\goalspace{\mathbb{G}}
\newcommand\startstate{s_{0}}
\newcommand\algo{\mathcal{A}}
\newcommand\approximateMDP{\hat{M}}
\newcommand\realMDP{M}
\newcommand\penalizedMDP{\tilde{M}}
\newcommand\penalizedcost{\tilde{c}}
\newcommand\incorrectset{\mathcal{X}}
\newcommand\deltamax{\Delta_{\mathsf{max}}}
\newcommand\tmp{\mathsf{tmp}}
\newcommand\buffer{\mathcal{D}}
\newcommand\reals{\mathbb{R}}
\newcommand\optimalpolicy{{\pi^*}}
\newcommand\vmax{V^{\mathsf{max}}}
\newcommand\covering{\mathcal{C}}

\newcommand\acmaxpp{\textsc{A-Cmax++}}
\newcommand\ecmax{\textsc{E-Cmax}}
\newcommand{\Mhat}{\hat{M}}
\newcommand\fhat{\hat{f}}
\newcommand\Mtilde{\tilde{M}}
\newcommand\ctilde{\tilde{c}}
\newcommand\best{\mathsf{best}}
\newcommand\Vhat{\hat{V}}
\newcommand\Vtilde{\tilde{V}}
\newcommand\loss{\mathcal{L}}
\newcommand\trainingset{\mathbb{X}}

\newcommand\xbold{\mathbf{x}}
\newcommand\ybold{\mathbf{y}}
\newcommand\betabold{\mathbf{\beta}}
\newcommand\wbold{\mathbf{w}}

%%% Local Variables:
%%% mode: latex
%%% TeX-master: "main"
%%% End:


% \draftstamp{\today}{DRAFT}

\begin {document}
\frontmatter

\pagestyle{empty}

\title{{\bf Provably Efficient Planning
    using Inaccurate Models by Learning from Experience}}
\author{Anirudh Vemula}
\date{Oct 2020}
\Year{2020}
%\trnumber{CMU-CS-19-109}

\committee{  
  \begin{tabular}{rl}
    & \\
Maxim Likhachev, Co-chair & \\
J. Andrew Bagnell, Co-chair & 
\end{tabular}
}

\support{}
\disclaimer{}

\keywords{}

\maketitle

% \begin{dedication}
%   To all of the people that light up my life. {\ensuremath\heartsuit}
% \end{dedication}

\begin{abstract}
  Modern planning methods are effective in computing feasible and
optimal plans for robotic tasks when given access to accurate
dynamical models.
However, robots operating in the real
world often face situations that cannot be modeled before
execution.
Thus, we only have access to simplified but potentially inaccurate 
models.
This imperfect modeling can lead to highly suboptimal plans
or even the inability to reach the goal during execution.
Existing approaches present a learning-based solution where real-world
experience is used to learn a complex dynamical model that is
subsequently used for planning. However, this requires a prohibitively
large amount of experience over the entire state space, and can
be wasteful if we are interested in completing the task and not in
modeling the dynamics accurately. Furthermore, for domains where
modeling the true dynamics is intractable such as deformable
manipulation, or vary over time due to 
wear and tear, the learned model can still end up being inaccurate.
This thesis argues that using a simplified and potentially inaccurate model for
planning allows us to significantly reduce the amount of real-world
experience needed to provably guarantee that the robot reaches the
goal.

In completed work, we proposed two approaches in support of this
argument. The first approach \cmax{} guarantees that the robot reaches the
goal using the inaccurate model without any resets. This
is achieved by biasing the planner away from transitions whose
dynamics are discovered to be inaccurately modeled during online
execution. However, \cmax{} requires strong assumptions on the
accuracy of the model used for planning and fails to improve the
quality of solution over repetitions of the same task. The second
approach \cmaxpp{} leverages real-world experience to improve the
quality of resulting plans over successive repetitions of a robotic
task. CMAX++ achieves this by integrating model-free learning using
acquired experience with model-based planning using the potentially
inaccurate model. In addition to completeness, \cmaxpp{} also
guarantees asymptotic convergence to the optimal path cost
as the number of repetitions increases under relaxed
assumptions. Crucially, both approaches do not require any updates to
the dynamics of the model unlike any previously existing method.
\end{abstract}

% \newgeometry{left=0.5in,right=0.5in,top=1in,bottom=1.4in}
% \begin{acknowledgments}
% \end{acknowledgments}
% \restoregeometry

\pagestyle{plain}

\tableofcontents
\addtocontents{toc}{\vspace*{-2cm}}
\listoffigures
\addtocontents{lof}{\vspace*{-2cm}}
\listoftables
\listofalgorithms

\mainmatter

\chapter*{Bibliography}
\addcontentsline{toc}{chapter}{Bibliography}

\vspace{-25mm}
This bibliography contains \total{citenum} references.
\vspace{10mm}

\printbibliography[heading=none]

\end{document}

%%% Local Variables:
%%% mode: latex
%%% TeX-master: t
%%% End:
